
\chapter{Anexos}

\begin{sidewaystable}
    \centering
    \caption{Síntesis de Estudios sobre la Solvatación del Ion Cu$^{2+}$ en Agua}
    \label{tab:agua-paramteros-estructurales-experimentales-y-teoricos}
    {\scriptsize % Abre el entorno para letra pequeña
    
    % --- Tabla Experimental ---
    \textbf{Estudios Experimentales}
    \vspace{2mm} % Un pequeño espacio vertical
    
    \begin{tabular}{@{}lllll@{}}
        \toprule
        \textbf{Referencia (Año)} & \textbf{Técnica} & \textbf{NC (Número de Coordinación)} & \textbf{Distancia Cu-O$_{eq}$ (\AA)} & \textbf{Distancia Cu-O$_{ax}$ (\AA)} \\
        \midrule
        Neilson et al. (1981)   & Difracción de neutrones & 6 $\pm$ 1                              & 1.97 $\pm$ 0.02                      & 2.60 $\pm$ 0.02                                 \\
        Sham et al. (1981)      & EXAFS                   & 4 + 2 (6-coordinado)                   & 1.96                                 & 2.60                                            \\
        Salmon et al. (1988)    & Difracción de neutrones & 4 + 2 (6-coordinado)                   & 1.96 $\pm$ 0.03                      & $\geq$ 2.21 (citado como 2.60)                  \\
        Pasquarello et al. (2001) & Difracción de neutrones y MD & 5-coordinado                           & ~1.95-2.0                            & -                                               \\
        Benfatto et al. (2002)  & EXAFS y XANES           & 5-coordinado (o 6-coordinado)          & 1.961 (para 6-coord)                 & 2.36 (para 6-coord)                             \\
        Frank et al. (2005)     & MXAN y XAS              & Principalmente 5-coordinado          & 1.95 - 1.98 $\pm$ 0.03               & 2.35 $\pm$ 0.05                                 \\
        Amira et al. (2005)     & CPMD                    & 5-coordinado                           & 2.00                                 & 2.45                                            \\
        Smirnov y Trostin (2009)& Revisión                & 6-coordinado                           & 1.96 $\pm$ 0.04                      & 2.40 $\pm$ 0.10                                 \\
        Bowron et al. (2013)    & Multi-técnica           & 4.5 $\pm$ 0.6 (promedio)               & -                                    & -                                               \\
        Frank et al. (2015)     & XAS (alta resolución)   & 5-coordinado (dominante)               & 1.95 - 1.97                          & 2.21 (5-coord), 2.19 y 2.33 (6-coord)           \\
        Frank et al. (2018)     & XAS y MXAN              & 6-coord (55\%) y 5-coord (43\%)        & 1.95 $\pm$ 0.01                      & 2.14 $\pm$ 0.06 y 2.28 $\pm$ 0.05 ("split axial") \\
        Persson et al. (2020)   & EXAFS                   & 6-coordinado (no centrosimétrico)      & 1.956 $\pm$ 0.003                    & 2.14 $\pm$ 0.02 y 2.32 $\pm$ 0.02 ("split axial") \\
        \bottomrule
    \end{tabular}

    \vspace{8mm} % Espacio mayor entre las dos tablas

    % --- Tabla Teórica ---
    \textbf{Estudios Teóricos}
    \vspace{2mm} % Un pequeño espacio vertical

    \begin{tabular}{@{}lllll@{}}
        \toprule
        \textbf{Referencia (Año)} & \textbf{Técnica} & \textbf{NC (Número de Coordinación)} & \textbf{Distancia Cu-O$_{eq}$ (\AA)} & \textbf{Distancia Cu-O$_{ax}$ (\AA)} \\
        \midrule
        Bérces et al. (1999)      & DFT y MD                & 4                                      & -                                    & -                                          \\
        Burda et al. (2004)       & DFT                     & 5 y 6-coordinados estables             & ~1.95-1.97 (5-coord), 1.956 (6-coord)  & 2.170 (5-coord), 2.281 (6-coord)           \\
        Pavelka y Burda (2005)    & DFT                     & 4 y 6-coordinados                      & 1.96 (4-coord), 1.98 (6-coord)       & 2.24 (6-coord)                             \\
        de Almeida et al. (2007)  & DFT (B3LYP)             & 6-coordinado (D2h)                     & 1.968 y 2.047 ("split")              & 2.296                                      \\
        Sukrat y Parasuk (2007)   & HF, MP2, B3LYP          & 5-coordinado (más estable)             & (~1.95)                              & < 2.29                                     \\
        Bryantsev et al. (2008)   & DFT y COSMO             & 5-coordinado (más estable en agua)     & 2.001                                & 2.273                                      \\
        O'Brien y Williams (2008) & Computacional (B3LYP)   & Consistente con 4 (contrib. 5 y 6)     & -                                    & -                                          \\
        Bryantsev et al. (2009)   & DFT y COSMO             & 5-coordinado (más estable)             & 2.001                                & 2.272                                      \\
        Rios-Font et al. (2010)   & DFT                     & 4 o 5-coordinados                      & -                                    & -                                          \\
        Liu et al. (2010)         & MD ab initio            & 5 y 6-coordinados (coexistencia)       & -                                    & -                                          \\
        Gómez-Salces et al. (2012) & Revisión y QM/MM       & 5-coordinación favorecida              & 2.03                                 & 2.15 y 2.30 (no centrosimétricos)          \\
        Galván-García et al. (2017) & DFT                   & Coexistencia de 4, 5, y 6              & 2.047-2.049 (6-coord, MP2)           & 2.215 (6-coord, MP2)                       \\
        Monjaraz-Rodríguez et al. (2018) & DFT              & 5 (domina) y 6                         & ~2.0                                 & hasta 2.5                                  \\
        Suzuki et al. (2019)      & 3D-RISM-SCF             & 5-6 (intercambiables)                  & 1.97-2.00 (Ci), 1.96-2.04 (D2h)      & 2.28 (Ci), 2.25 (D2h)                      \\
        Christensen y Steele (2023) & Química cuántica ab initio & Coexistencia de 4, 5, y 6       & -                                    & -                                          \\
        Da-yang et al. (2024)     & MP2/IEF-PCM             & 4, 5, y 6 (para n=10)                  & 1.99 (5-coord), 2.01 (6-coord)       & 2.19 (5-coord), 2.26 (6-coord)             \\
        \bottomrule
    \end{tabular}
    
    } % Cierra el entorno de letra pequeña
\end{sidewaystable}




\begin{sidewaystable}
    \centering
    \caption{Síntesis de Estudios sobre la Solvatación del Ion Cu$^{2+}$ en Metanol}
    \label{tab:metanol-paramteros-estructurales-experimentales-y-teoricos}
    {\scriptsize % Abre el entorno para letra scriptsize
    
    % --- Tabla Experimental ---
    \textbf{Estudios Experimentales}
    \vspace{2mm} % Un pequeño espacio vertical
    
    \begin{tabular}{@{}lllll@{}}
        \toprule
        \textbf{Referencia (Año)} & \textbf{Técnica} & \textbf{NC (Número de Coordinación)} & \textbf{Distancia Cu-O$_{eq}$ (\AA)} & \textbf{Distancia Cu-O$_{ax}$ (\AA)} \\
        \midrule
        Ichikawa y Kevan (1980)   & Análisis de eco de espín electrónico & 6                                         & -                                 & 2.1                                    \\
        Helm et al. (1986)        & Oxygen-17 NMR                        & 6 (transitan a 5)                         & -                                & -                                      \\
        Inada et al. (1999)       & EXAFS                                & 3.8                                      & 1.97                             & -                                     \\
        Funahashi y Inada (2002)  & EXAFS                                & -                                        & 2.02                             & -                                     \\
        Zitolo et al. (2012)      & XAS (EXAFS/XANES)                    & 5 (según XANES)                          & 1.96 (EXAFS), 1.95 (XANES)  & 2.28 (EXAFS), 2.23 (XANES)   \\
        Smirnov (2013)            & Revisión comparativa                 & -                                        & -                                & -                                     \\
        Persson et al. (2020)     & EXAFS                                & 6 (octaédrica no centrosimétrica)      & 1.975(3) (media)                 & 2.202(8) y 2.34(1)                \\
        \bottomrule
    \end{tabular}

    \vspace{8mm} % Espacio mayor entre las dos tablas

    % --- Tabla Teórica ---
    \textbf{Estudios Teóricos}
    \vspace{2mm} % Un pequeño espacio vertical

    \begin{tabular}{@{}lllll@{}}
        \toprule
        \textbf{Referencia (Año)} & \textbf{Técnica} & \textbf{NC (Número de Coordinación)} & \textbf{Distancia Cu-O$_{eq}$ (\AA)} & \textbf{Distancia Cu-O$_{ax}$ (\AA)} \\
        \midrule
        Tabouli E. D. et al. (2022) & MP2 (fase gas)          & 4 o 6 (depende de T)  & 1.94 (NC=4), 1.99 (NC=5), 2.01 (NC=6)  & 2.21 (NC=5), 2.27 (NC=6)  \\
        Tabouli E. D. et al. (2023) & M06-2X (fase gas)       & 5 y 6 favorecidos     & 1.95 (NC=4), 2.00 (NC=5), 2.02 (NC=6)  & 2.15 (NC=5), 2.26 (NC=6)  \\
        Tabouli E. D. et al. (2023) & M06-2X + IEF-PCM        & 6 y 5 dominan         & 1.97 (NC=4), 1.99 (NC=5), 2.02 (NC=6)  & 2.16 (NC=5), 2.24 (NC=6)  \\
        \bottomrule
    \end{tabular}
    
    } % Cierra el entorno de letra scriptsize
\end{sidewaystable}

