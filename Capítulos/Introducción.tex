\chapter{Introducción}
    \section{Justificación} 
        La investigación teórica sobre la solvatación del ion cobre \( \text{Cu}^{2+} \) en agua y metanol es de gran relevancia debido a su importancia en diversos procesos químicos e industriales. El cobre es un elemento esencial en la catálisis, la electroquímica y la bioquímica, y su interacción con solventes como el agua y el metanol afecta directamente su comportamiento químico y su reactividad. %
        En este contexto, el análisis de la solvatación de \( \text{Cu}^{2+} \) en estos solventes contribuye al desarrollo de modelos más precisos y al avance del conocimiento en el campo de la química teórica y computacional. \cite{lozano2014que} %
    
    \section{Objetivos}
        \subsection{Objetivo General}
        \subsection{Objetivos particulares}