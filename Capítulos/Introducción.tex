\chapter{Introducción a la Teoría de la Solvatación}

\section{La Relevancia del Solvente en Procesos Químicos y Biológicos}

El estudio de los sistemas químicos ha estado históricamente centrado en la estructura electrónica de las moléculas, ya que esta dicta de manera fundamental su comportamiento \cite[102]{hirata2003molecular}. Sin embargo, la gran mayoría de los procesos químicos y biológicos no ocurren en el vacío, sino en un medio líquido, comúnmente una disolución \cite[2]{canuto2010solvation}. En este contexto, la influencia del solvente no es un efecto menor o una simple perturbación; es un factor determinante que puede gobernar la dirección, la eficiencia y la velocidad de una reacción \cite[72]{hirata2003molecular}, \cite[4]{nishiyama2021molecular}. La energía asociada a la solvatación es, en muchos casos, de una magnitud comparable a la de los cambios en la energía electrónica de los reactivos, lo que subraya su papel central \cite[10]{hirata2003molecular}. Por tanto, una descripción teórica completa de cualquier proceso en disolución debe incluir de manera explícita los efectos del solvente \cite[1]{nishiyama2021molecular}.

Esta influencia es particularmente crítica en el campo de la bioquímica. La estabilidad conformacional de biomoléculas como las proteínas y el ADN está dominada por las interacciones con el solvente acuoso circundante \cite[6, 116]{hirata2003molecular}. Fenómenos como la desnaturalización de proteínas a bajas temperaturas no pueden ser explicados sin una comprensión detallada del rol del agua y los efectos hidrofóbicos \cite[12]{hirata2003molecular}. Del mismo modo, la función de muchas proteínas depende de la presencia de iones y de cómo estos interactúan con sus capas de solvatación \cite[2]{marcus2015ions}. La comprensión de la solvatación, por lo tanto, es indispensable no solo para la química fundamental, sino también para el diseño de fármacos, la ciencia de materiales y la biología molecular.

\section{Fenómenos Físico-Químicos Dependientes del Entorno Solvatado}

La criticidad del entorno solvatado se manifiesta en una amplia gama de fenómenos. Las reacciones químicas y la estabilidad conformacional de proteínas son dos de los ejemplos más importantes en la química teórica, donde el solvente puede incluso revertir el equilibrio entre reactivos y productos \cite[8, 102]{hirata2003molecular}. Procesos fundamentales como los equilibrios ácido-base y la autoionización del agua están intrínsecamente ligados a la capacidad del solvente para estabilizar especies cargadas \cite[86, 89]{hirata2003molecular}.

Otros ejemplos notables incluyen:
\begin{itemize}
    \item \textbf{Reacciones de Transferencia de Carga:} Procesos como las reacciones S\textsubscript{N}2 y la transferencia de electrones (ET) son extremadamente sensibles al medio, ya que la reorganización del solvente es un componente clave de la barrera de activación energética \cite[98, 102]{hirata2003molecular}, \cite[4]{nishiyama2021molecular}.
    \item \textbf{Equilibrios Tautoméricos:} El equilibrio ceto-enol, por ejemplo, se desplaza significativamente hacia la forma ceto en solventes de alta polaridad debido a la estabilización diferencial de los tautómeros \cite[93]{hirata2003molecular}.
    \item \textbf{Procesos Espectroscópicos:} El color de una molécula de colorante, un fenómeno conocido como solvatocromismo, puede cambiar drásticamente según el solvente, debido a cómo este estabiliza los estados electrónicos fundamental y excitado de la molécula \cite[12]{hirata2003molecular}, \cite[2]{canuto2010solvation}.
    \item \textbf{Efectos Iónicos Específicos:} La adición de sales al agua puede alterar drásticamente la estabilidad y solubilidad de proteínas, un fenómeno conocido como la serie de Hofmeister, que depende de la identidad específica del ion y su interacción con el agua \cite[146]{hirata2003molecular}, \cite[3]{marcus2015ions}.
\end{itemize}

\section{Naturaleza de las Interacciones Intermoleculares en Disolución}

La solvatación es el resultado macroscópico de un balance delicado de interacciones a nivel molecular. Estas interacciones pueden ser clasificadas en varias categorías. Las \textbf{fuerzas electrostáticas} de largo alcance, de naturaleza Coulombica, son fundamentales para describir la interacción entre iones y moléculas polares \cite[8]{hirata2003molecular}. En un medio denso, esta interacción se ve modificada por efectos de apantallamiento dieléctrico por parte del solvente \cite[26]{hirata2003molecular}.

A distancias más cortas, las interacciones de \textbf{van der Waals} se vuelven dominantes, consistiendo en una repulsión de corto alcance debida al principio de exclusión de Pauli y una atracción de rango intermedio (fuerzas de dispersión de London) \cite[57]{hirata2003molecular}, \cite[3]{canuto2010solvation}. Comúnmente, estas interacciones se modelan mediante potenciales sitio-sitio, como el potencial de Lennard-Jones (12-6), que se suma al término de Coulomb para describir la interacción total \cite[62, 106]{hirata2003molecular}.

En solventes próticos como el agua o el metanol, el \textbf{enlace de hidrógeno} emerge como una interacción direccional y de corto alcance de importancia capital. Es responsable de las propiedades anómalas del agua y es crucial para la estructura y función de las biomoléculas \cite[10]{hirata2003molecular}. Un tratamiento teórico preciso del enlace de hidrógeno es uno de los mayores desafíos para los modelos de solvatación \cite[5]{canuto2010solvation}. Finalmente, los \textbf{efectos de inducción} describen la polarización de la nube electrónica de una molécula debido al campo eléctrico generado por las moléculas vecinas, lo que constituye un efecto de muchos cuerpos no aditivo \cite[4]{canuto2010solvation}.

\section{Estructura y Organización en la Capa de Solvatación}

La manifestación estructural de estas interacciones es la formación de una región de solvente altamente organizada alrededor del soluto, conocida como la \textbf{capa de solvatación} o esfera de coordinación \cite[1]{canuto2010solvation}. La caracterización de esta estructura local es fundamental para entender la reactividad. La herramienta teórica y experimental más importante para este fin es la \textbf{Función de Distribución Radial (RDF)}, denotada como $g(r)$, que describe la probabilidad de encontrar una partícula a una distancia $r$ de una partícula de referencia \cite[13, 22]{hirata2003molecular}, \cite[7]{nishiyama2021molecular}. Los picos en la RDF corresponden a las capas de solvatación ordenadas, y su integración conduce al \textbf{número de coordinación}, que cuantifica el número de moléculas de solvente en la primera capa \cite[8]{canuto2010solvation}, \cite[18]{nishiyama2021molecular}.

Mientras que la RDF proporciona información sobre la distancia, la \textbf{Función de Distribución Angular (ADF)} ofrece información sobre la orientación de las moléculas de solvente, lo cual es esencial para caracterizar interacciones direccionales como los enlaces de hidrógeno \cite[279]{hirata2003molecular}. Juntas, estas funciones permiten construir una imagen tridimensional detallada de la microestructura del solvente alrededor del soluto.

\section{Modelos Teóricos para el Estudio de la Solvatación}

El tratamiento teórico de la solvatación ha evolucionado desde modelos macroscópicos hasta descripciones moleculares explícitas. Históricamente, los \textbf{modelos de continuo} han sido la piedra angular, representando al solvente como un medio dieléctrico continuo caracterizado por su permitividad estática, $\epsilon$ \cite[8]{hirata2003molecular}. La ecuación de Born para la energía libre de solvatación de un ion es un ejemplo arquetípico \cite[6]{nishiyama2021molecular}. Aunque son computacionalmente eficientes, estos modelos fallan en capturar la naturaleza molecular del solvente, omitiendo efectos como el empaquetamiento, los enlaces de hidrógeno y la estructura de capas \cite[5, 23]{canuto2010solvation}.

Para superar estas limitaciones, se desarrollaron métodos basados en la \textbf{simulación molecular}, como la Dinámica Molecular (MD) y los métodos de Monte Carlo (MC) \cite[9]{hirata2003molecular}. Estos enfoques tratan tanto al soluto como a las moléculas de solvente de forma explícita, permitiendo un análisis detallado de la estructura y la dinámica. La MD, en particular, resuelve las ecuaciones de movimiento de Newton para un conjunto de partículas que interactúan a través de un campo de fuerza, generando trayectorias que representan la evolución temporal del sistema \cite[22]{nishiyama2021molecular}.

Un tercer enfoque combina la precisión de la mecánica cuántica con la eficiencia de los modelos clásicos: los \textbf{métodos híbridos QM/MM} (Quantum Mechanics/Molecular Mechanics). En este formalismo, la región de interés (el soluto y su primera capa de solvatación) se trata con un método de alto nivel de química cuántica (QM), mientras que el resto del solvente se describe con un campo de fuerza clásico (MM) \cite[15]{canuto2010solvation}. Este enfoque permite estudiar procesos que involucran cambios en la estructura electrónica, como reacciones químicas y fenómenos espectroscópicos, dentro de un entorno solvatado realista \cite[10]{nishiyama2021molecular}. La combinación de estos métodos computacionales proporciona las herramientas necesarias para abordar la complejidad de los sistemas en disolución, un paso esencial para el avance de la química, la biología y la ciencia de materiales.

%%%%%%%%%

\section{Relevancia Biológica e Industrial del Ion Cúprico}

El estudio de la solvatación iónica adquiere una dimensión particular cuando se enfoca en cationes de metales de transición, debido a su papel ubicuo y fundamental en una vasta gama de procesos naturales y tecnológicos. Dentro de este grupo, el ion cobre(II) (\ce{Cu^{2+}}) emerge como un sistema de extraordinario interés, no solo por su complejidad inherente, sino también por su profunda implicación en la bioquímica y la industria \cite{Wa-2012-01}. El cobre es el tercer metal más abundante en el cuerpo humano, superado únicamente por el hierro y el zinc \cite{Me-2012-01}, y su correcta regulación es vital para la salud. La desregulación de la homeostasis del cobre, por ejemplo, ha sido directamente relacionada con el desarrollo de severas enfermedades neurodegenerativas, incluyendo las de Alzheimer y Parkinson \cite{Cu-2015-01, Me-2022-01, Wa-2018-01, Wa-2024-03}.

La relevancia biológica del \ce{Cu^{2+}} radica en su versatilidad redox, que le permite participar como cofactor esencial en una multitud de metaloenzimas. Estas proteínas están involucradas en procesos críticos como el transporte de oxígeno, la transferencia de electrones en la cadena respiratoria y la catálisis de reacciones de óxido-reducción \cite{Cu-2015-01, Me-2022-01, Wa-2009-01, Wa-2016-01}. Además, se ha demostrado que el ion \ce{Cu^{2+}} puede catalizar la formación de enlaces peptídicos, pero su presencia incontrolada también puede inducir la agregación y el desplegamiento de proteínas, generando especies reactivas de oxígeno (ROS) con un alto potencial tóxico \cite{Wa-2009-01, Wa-2016-01}. Esta dualidad entre funcionalidad esencial y toxicidad potencial hace que la comprensión de su comportamiento en el entorno biológico—predominantemente acuoso—sea un objetivo primordial de la química bioinorgánica \cite{Wa-2018-01}.

\section{El Desafío de la Hidratación del \ce{Cu^{2+}}: El Efecto Jahn-Teller}

La aparente simplicidad del ion \ce{Cu^{2+}} en agua desmiente una complejidad estructural que ha sido objeto de un intenso y prolongado debate científico \cite{Cu-2011-01, Wa-2002-01}. La fuente de esta complejidad reside en su configuración electrónica $d^9$, la cual, en un campo de ligandos de simetría octaédrica, es susceptible a la distorsión de Jahn-Teller \cite{Cu-2011-01, Wa-2001-01, Cu-2004-01, Wa-2008-02, Wa-2009-02, Wa-2020-01}. Este efecto rompe la degeneración de los orbitales $e_g$, resultando típicamente en una elongación de los enlaces axiales del complejo hexacoordinado $[\ce{Cu(H2O)6}]^{2+}$, una imagen que fue aceptada durante muchos años como el modelo estándar \cite{Wa-2007-02, Wa-2018-02}.

Sin embargo, esta visión ha sido desafiada por una creciente cantidad de evidencia teórica y experimental que sugiere un panorama más dinámico y heterogéneo. Lejos de ser un octaedro estático, la primera esfera de solvatación del \ce{Cu^{2+}} acuoso es altamente plástica \cite{Wa-2018-02}. Se ha propuesto que la barrera energética para la conversión entre diferentes geometrías es muy baja, permitiendo la coexistencia de múltiples especies en equilibrio dinámico. Las simulaciones de dinámica molecular ab initio (AIMD) han revelado que el ion puede fluctuar rápidamente entre geometrías de pirámide cuadrada (pentacoordinada) y bipirámide trigonal \cite{Wa-2001-01, Wa-2010-03}. Esta plasticidad estructural es la razón por la cual diferentes técnicas experimentales han arrojado resultados aparentemente contradictorios sobre el número de coordinación y la geometría preferida del ion.

\section{Análisis Comparativo de la Estructura de Solvatación}

Para validar un nuevo método computacional, como el que se propone en este trabajo, es indispensable contrastar sus resultados con el cuerpo de conocimiento existente, que se ha construido a partir de una combinación de técnicas experimentales y teóricas. A continuación, se presenta una síntesis de los parámetros estructurales reportados en la literatura para el ion \ce{Cu^{2+}} en disolución acuosa y metanólica.

\subsection{Número de Coordinación: Un Consenso Emergente}

El número de coordinación (NC) del ion \ce{Cu^{2+}} ha sido el foco principal del debate. Mientras que los cationes divalentes de tamaño similar, como el \ce{Ni^{2+}}, exhiben una clara coordinación octaédrica (NC=6), el \ce{Cu^{2+}} es anómalo \cite{Wa-1982-01, Wa-2001-01}.

En **solución acuosa**, los primeros estudios de difracción de neutrones ya apuntaban a una coordinación no convencional \cite{Wa-1982-01}. Trabajos más recientes, que combinan múltiples técnicas experimentales como la espectroscopía de absorción de rayos X (XAS, EXAFS, XANES) y la dispersión de neutrones, junto con simulaciones teóricas, han consolidado la idea de que la coordinación no es estrictamente seis. En su lugar, predomina una visión de una coordinación promedio que fluctúa. Varios estudios concluyen que la especie pentacoordinada $[\ce{Cu(H2O)5}]^{2+}$ es la más prevalente o energéticamente favorable en disolución \cite{Wa-2002-01, Wa-2005-01, Wa-2005-02, Wa-2008-01, Wa-2010-03, Wa-2012-01, Wa-2024-01}. Sin embargo, también hay fuerte evidencia de la coexistencia de especies tetracoordinadas y hexacoordinadas \cite{Wa-2009-01, Wa-2017-01, Wa-2023-01}. En un estudio de 2018, se sugiere que los isómeros pentacoordinados son los más comunes, pero la contribución de los hexacoordinados aumenta con el número de moléculas de agua disponibles en el clúster \cite{Wa-2018-01}.

En **metanol**, la evidencia es más escasa, lo que justifica la realización de nuevos estudios computacionales. Un estudio experimental de XAS concluyó inequívocamente que el NC promedio es 5 \cite{Me-2012-01}. No obstante, otros trabajos reportan valores diferentes, como 3.8 \cite{Me-2023-01}. Los estudios teóricos más recientes sugieren que, de manera similar al agua, coexisten múltiples estados de coordinación, con una preferencia por las geometrías pentacoordinada y hexacoordinada \cite{Me-2022-01, Me-2023-02}.

La siguiente tabla resume los hallazgos de varios estudios clave sobre el número de coordinación.

\begin{table}[h!]
\centering
\caption{Número de Coordinación (NC) del ion \ce{Cu^{2+}} reportado en la literatura.}
\label{tab:coordination_number}
\begin{tabular}{@{}llll@{}}
\toprule
Solvente & NC Reportado & Método & Referencia \\ \midrule
Agua     & $4.1 \pm 0.3$ a $4.5 \pm 0.6$ & ND, XRD, EXAFS, EPSR & \cite{Wa-2013-01} \\
Agua     & Predominantemente 5 (90\%)   & CPMD                      & \cite{Wa-2005-02} \\
Agua     & 5 o 6 (fluctuante)           & AIMD                      & \cite{Wa-2004-02} \\
Agua     & 6 (primera capa)             & QM/MM (MP2)               & \cite{Cu-2004-01} \\
Agua     & 5 es el más estable          & DFT/COSMO                 & \cite{Wa-2008-01} \\
Agua     & 5 y 6 (coexistencia)         & 3D-RISM-SCF               & \cite{Wa-2019-01} \\
Metanol  & 5 (inequívocamente)          & XAS (EXAFS, XANES)        & \cite{Me-2012-01} \\
Metanol  & 3.8                          & EXAFS                     & \cite{Me-2023-01} \\
Metanol  & 6 (predominante)             & EXAFS                     & \cite{Wa-2020-01} \\
Metanol  & 5 y 6 son relevantes         & DFT                       & \cite{Me-2022-01} \\ \bottomrule
\end{tabular}
\end{table}

\subsection{Geometría de Coordinación y Distancias de Enlace}

La geometría de los complejos de solvatación del \ce{Cu^{2+}} está directamente ligada a su número de coordinación y al efecto Jahn-Teller. Para las especies acuosas, la discusión se ha centrado en distinguir entre un octaedro distorsionado (simetría $D_{4h}$), una pirámide cuadrada ($C_{4v}$) y una bipirámide trigonal ($D_{3h}$) \cite{Wa-2001-01, Wa-2010-03}.

Los resultados más recientes favorecen una imagen donde la **pirámide cuadrada**, a menudo distorsionada, es una estructura central, ya sea como una especie estable de cinco ligandos o como un intermediario en la dinámica de un complejo de seis ligandos \cite{Wa-2002-01, Wa-2005-02, Wa-2008-01, Me-2012-01}. La evidencia experimental de alta resolución ha descartado la existencia de un octaedro elongado simétrico, favoreciendo en su lugar geometrías asimétricas y no centrosimétricas \cite{Wa-2015-01, Wa-2018-02}.

Un aspecto clave que caracteriza estas geometrías es la diferencia entre las distancias de enlace axiales y ecuatoriales. El efecto Jahn-Teller típicamente induce una elongación de los enlaces axiales. La Tabla \ref{tab:distances} recopila algunas de las distancias \ce{Cu-O} reportadas, mostrando una notable consistencia entre diferentes métodos. Las distancias ecuatoriales se agrupan consistentemente alrededor de \SI{1.95}{\angstrom} a \SI{2.00}{\angstrom}, mientras que las distancias axiales son más largas y variables, típicamente entre \SI{2.20}{\angstrom} y \SI{2.40}{\angstrom}. Esta diferencia es la huella digital de la distorsión electrónica del ion.

\begin{table}[h!]
\centering
\caption{Distancias de enlace \ce{Cu-O} reportadas en la literatura para la primera esfera de solvatación.}
\label{tab:distances}
\begin{tabular}{@{}lllll@{}}
\toprule
Solvente & \ce{Cu-O_{eq}} (\AA)           & \ce{Cu-O_{ax}} (\AA)           & Método          & Referencia \\ \midrule
Agua     & $1.96 \pm 0.03$                & $\sim 2.3 - 2.4$               & ND, XRD         & \cite{Wa-1988-01, Wa-2005-01} \\
Agua     & $2.03$                         & $2.30$                         & QM/MM (HF)      & \cite{Wa-2003-01} \\
Agua     & $2.07$                         & $2.35$                         & QM/MM (MP2)     & \cite{Cu-2004-01} \\
Agua     & $2.00$                         & $2.45$                         & CPMD (BLYP)     & \cite{Wa-2005-02} \\
Agua     & $\sim 1.98$                    & $\sim 2.16$                    & DFT (BHLYP)     & \cite{Wa-2007-02} \\
Metanol  & $1.96$                         & $2.28$  / $2.32$               & XAS             & \cite{Me-2012-01} \\
Metanol  & $\sim 1.96$                    & $\sim 2.15 - 2.32$             & EXAFS           & \cite{Wa-2020-01} \\ \bottomrule
\end{tabular}
\end{table}

\subsection{Métodos Teóricos y su Evolución}

La complejidad del ion \ce{Cu^{2+}} ha impulsado el desarrollo y la aplicación de métodos teóricos cada vez más sofisticados. Las simulaciones clásicas con campos de fuerza estándar fracasan en describir la distorsión de Jahn-Teller sin correcciones ad hoc \cite{Wa-2008-01}. Por ello, los enfoques basados en la mecánica cuántica se han vuelto indispensables.

Los estudios iniciales emplearon métodos de clúster en fase gaseosa, pero pronto se hizo evidente que para una descripción realista era necesario incluir los efectos del solvente a granel, ya sea implícitamente a través de modelos de continuo polarizable (PCM, COSMO) o explícitamente \cite{Wa-2008-01, Wa-2024-01}. Los modelos mixtos, que tratan un número definido de moléculas de solvente de manera explícita dentro de un campo de reacción de continuo, han demostrado ser una estrategia eficaz \cite{Wa-2009-01}.

La **Teoría del Funcional de la Densidad (DFT)**, especialmente en el marco de simulaciones de dinámica molecular ab initio (AIMD) como Car-Parrinello (CPMD) o Born-Oppenheimer (BOMD), se ha consolidado como la herramienta principal. La elección del funcional de intercambio-correlación es crítica. Funcionales como BLYP han sido utilizados con éxito para describir la dinámica de sistemas acuosos \cite{Wa-2004-02, Wa-2005-02}, mientras que funcionales híbridos como B3LYP o PBE0, que incluyen una porción de intercambio exacto de Hartree-Fock, han demostrado ser superiores para describir las propiedades estructurales y energéticas de los complejos de cobre \cite{Wa-2003-01, Wa-2009-01, Wa-2010-02}. Funcionales más modernos como M06-2X también han sido validados por su buen rendimiento en la descripción de interacciones no covalentes, cruciales en la segunda esfera de solvatación \cite{Me-2022-02, Me-2023-01}.

Finalmente, los métodos de teoría de perturbaciones de Møller-Plesset de segundo orden (MP2) y enfoques más avanzados como CCSD(T), aunque computacionalmente muy demandantes, sirven como un "estándar de oro" para calibrar la precisión de los métodos DFT \cite{Wa-2017-01}. La concordancia entre los resultados de DFT y estos métodos de alto nivel valida el uso de los primeros para sistemas de mayor tamaño y escalas de tiempo más largas, como las que se abordan en la presente investigación. La sinergia entre los avances en la teoría, los algoritmos y la potencia del supercómputo ha sido, por tanto, indispensable para alcanzar el nivel de comprensión actual sobre este fascinante ion.
