\documentclass[letterpaper,12pt,oneside]{book}
\usepackage[top=1in, left=0.9in, right=1.25in, bottom=1in]{geometry}
\usepackage{bachelorstitlepageUNAM}
%-------------------------------
\author{Jorge Angel Rosas Martínez}
\title{Caracterización teórica de la solvatación del dicatión de cobre en agua y metanol mediante simulaciones de dinámica molecular}
\faculty{Facultad de Química}
\degree{Licenciado en Ingeniería Química}
\supervisor{Dr. César Iván León Pimentel}
\cityandyear{Ciudad Universitaria, CDMX, 2024}
\logouni{Escudo-UNAM.pdf}
\logofac{logofq.jpeg}

%---------------------------------
\usepackage[T1]{fontenc}
\usepackage[utf8]{inputenc}
\usepackage[spanish,es-nodecimaldot,es-tabla]{babel}
\usepackage{graphicx}
\usepackage{tikz} 
\usepackage{tocloft}
\graphicspath{{./figs/}}
\usepackage{setspace}
%\usepackage{code}
%\renewcommand\cftsecpresnum{\S}
%\renewcommand\cftsubsecpresnum{\S}   

\begin{document}
\frontmatter
\maketitle
\chapter*{}
\begin{flushright}%
  \emph{Para mis padres Jorge Rosas y Luz María 1 } 
  \thispagestyle{empty}
\end{flushright}

\chapter{Agradecimientos}
\spacing{1.5}%\doublespacing
En esta página escribiré los agradecimientos correspondientes, pero sin lugar a dudas entre todas las personas irán:

Mi padres Jorge Rosas y Luz María, mis dos hermanos.

Mi mejor amigo de toda la vida, el ingeniero Osvaldo González 

\chapter{Introducción}

\tableofcontents
\listoffigures

    
\mainmatter

\chapter{Hidrodinámica relativista} 

En este capítulo se utiliza información de estudios previos

\bibliographystyle{humannat}
\bibliography{references}

\backmatter%@sglvgdor


\end{document}